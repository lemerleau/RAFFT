% Created 2021-03-28 dim. 21:56
% Intended LaTeX compiler: pdflatex
\documentclass[a4paper,12pt]{article}
\usepackage[position=top,labelformat=empty]{subfig}
\usepackage{caption}
\usepackage[hmargin=2cm,vmargin=3cm]{geometry}


\usepackage{amsmath}
\usepackage{caption,graphicx,subcaption}
\usepackage[boxed]{algorithm2e}
\usepackage{authblk}
\author[1]{Vaitea Opuu}
\author[1]{Nono S. C. Merleau}
\author[1]{Matteo Smerlak}
\affil[1]{Max Planck Institute for Mathematics in the Sciences, D-04103 Leipzig, Germany}
\date{\today}
\title{A Mirror encoding combined with the FFT for a fast heuristic of the RNA folding dynamics}
\begin{document}

\maketitle

\section{Abstract}
\label{sec:org1da457b}
\begin{itemize}
\item Simple and fast heuristic for the folding path of RNAs.
\item It is straightforward to model Pseudoknots
\item It's performance is comparable to exact method on the RNA folding problem
\item It follows a simple idea which naively corresponds to RNA folds mechanism
(many BPs formed at once to compensate for the lost of entropy)
\item Among the 50 predicted structures, in average, at least one has pvv \textasciitilde{} 74\% and
sensitivity \textasciitilde{} 76\%.
\item We propose a fast algorithm method based on the FFT to search for high density
BP regions.
\item There are smooth coarse-grain folding path which lead to near native structures.
\end{itemize}

\clearpage
\section{Introduction}
\label{sec:orgd78e655}
\subsection{RNA folding introduction}
\label{sec:org46c0b3b}
bla bla dynamic of secondary structure relevant bla biological function.

\begin{itemize}
\item MFE and MEA not significantly different in term of performances (how to bench RNA)
\end{itemize}

\subsection{RNA folding dynamics}
\label{sec:org1e69860}
\begin{enumerate}
\item Description of RNA structure
\item going up to the 2ndary structure only
\item Simple rules to compute a structure: multiple BPs compensate the lost of
entropy during the folding process.
\end{enumerate}
\subsection{Energy model}
\label{sec:orgf036e64}
\begin{enumerate}
\item issue with additivity principle in model. Might be worst when the sequence
lengthens since more tertiary interactions interplay.
\end{enumerate}
\subsection{Existing methods}
\label{sec:org9407dfa}
\begin{enumerate}
\item MC sampling: kinefold; atomic moves; MC-style simulation
\item Barrier trees from conformation landscape subopt tree: Sample from the
boltzmann ensemble of structures
\item Vfold, simplified folding model
\end{enumerate}

\clearpage
\section{FFT based folding dynamic heuristic}
\label{sec:org4a0ce0a}
We now describe the heuristic folding algorithm starting from one sequence S and
its associated unfolded structure of lenght L. We first create a numerical
representation of S where each type of nucleotide in replaced by a unit vector
of 4 components:
\begin{equation}
\begin{split}
A \rightarrow \begin{pmatrix} 1 0 0 0 \end{pmatrix}
U \rightarrow \begin{pmatrix} 0 0 0 1 \end{pmatrix}
C \rightarrow \begin{pmatrix} 0 1 0 0 \end{pmatrix}
G \rightarrow \begin{pmatrix} 0 0 1 0 \end{pmatrix}
\end{split}
\end{equation}
which gives us a \(4 \times L\) matrix we call X where each row is a nucleotide
type channel. Here, the first row would be the A channel which we refer to as
\(X^A\). Then, we create a second copy for which we revert the order of the
sequence and use the following complementary encoding:
\begin{equation}
\begin{split}
\bar{A} \rightarrow \begin{pmatrix} 0 0 0 w_{\scalebox{0.5}{AU}} \end{pmatrix}
\bar{U} \rightarrow \begin{pmatrix} w_{\scalebox{0.5}{AU}} w_{\scalebox{0.5}{GU}} 0 0 \end{pmatrix}
\bar{C} \rightarrow \begin{pmatrix} 0 0 w_{\scalebox{0.5}{GC}} 0 \end{pmatrix}
\bar{G} \rightarrow \begin{pmatrix} 0 w_{\scalebox{0.5}{GC}} 0 w_{\scalebox{0.5}{GU}} \end{pmatrix}
\end{split}
\end{equation}
where \(w_{AU}\), \(w_{GC}\), \(w_{GU\) are tunable parameters for the next step. We
call this new copy \bar{X}, the mirror of X.

For each of the 4 components, we compute the correlation between X and \bar{X}
and simply sum up the four channels to obtain the correlation between the two
copies:
\begin{equation}
cor(k) = (c_{X^A,\bar{X}^A}(k) + c_{X^U,\bar{X}^U}(k) + c_{X^G,\bar{X}^G}(k) + c_{X^C,\bar{X}^C}(k)) / min(k, 2 \times L-k)
\end{equation}
where \(c_{X^A,\bar{X}^A(k)\) is the correlation in the \(A\) channel between the
two copies. The correlation \(cor(k)\) gives the average number of base pairs for
a positional lag \(k\). One channel correlation between the copies is given by:
\begin{equation}
c_{X^A,\bar{X}^A}(k) = \sum\limits_{1\leq i \leq L, 1 \leq i + k \leq M} X^A(i) \times \bar{X}^A(i+k)
\end{equation}
where \(X^A(i)\) and \(\bar{X}^A(i+k)\) are the A channel of site \(i\) and \(i+k\).
X\textsuperscript{A}(i) \texttimes{} \bar{X}\textsuperscript{A}(i+k) is non zero if sites \(i\) and \(i+k\) can form a base
pair, and will be the value of the chosen weight as described above. Although
this operation requires \(O(N^2)\) operation, it can take advantage of the FFT
which reduces drastically its complexity to \(O(Nlog(N))\).

The large correlation values between the two copies indicates the positional lag
between at which the base pair density is high. Therefore, we use a sliding
window strategy to search for the longest consecutive base pairs within the
positional lag. Since the copies are symmetrical, we only need to slide over one
half of the positional lag. Once the longest base pairs are identified, we
simply compute the free energy change when those base pair are formed. We
perform the same search for the \(n\) highest correlation lags, which gives us \(n\)
possible possibilities. Then, we added to the current structure the base pairs
that gives the best change of energy.

We are now left with two segments, the interior and exterior of the group of
consecutive base pairs formed. The two exterior fragments are concatenated
together. Then, we simply apply recursively the same procedure on the two
segments separately in a "Breath First" fashion to form new consecutive base
pairs, until no base pair formation can improve the energy. However, it is
straightforward to consider pseudoknots by simply concatenating all the
fragments left.

The algorithm described so far tends to be stuck in the first local minima found
along the folding trajectory. To alleviate this, we propose a stacking procedure
where the 50 best trajectories are stored in a stack and evolved in parallel.
Hence, it offers the flexibility of overcoming some energy barriers. \textbf{Figure}
shows the whole procedure.

\section{Folding RNAs}
\label{sec:orgb7ca96f}
To evaluate the relevance of the folding dynamic heuristic, we compared the
algorithm performance for the folding task. In addition, to assess the effect of
sequence lengthens on these predictions, we analyzed their performance
length-wise.

\textbf{Figure} shows the performance in PPV and sensitivity for the four methods. It
shows that the ML method is consistently better than thermodynamic methods.
Length-wise T-test between the MFE and ML predicitons showed that this
difference is significant (pvalue \(\approx\) 10\textsuperscript{-12}) with a substantial
improvement of about 10\%. Although RAFFT predictions were found to be comparable
to MFE predictions, they are significantly less accurate (pvalue \(\approx\)
0.0002), with a drastic lost of performance for sequences of length greater than
300 nucleotides.

Among the 50 configurations produced by RAFFT, we found in average at least one
prediction with in average 59\% of PPV and <SENS> of sensitivity (blue curve in
\textbf{figure}). The overall gain of performances is not significantly different from
the MFE predictions. However, for the sequences of length lesser than 200
nucleotides, this gain was found to be substantial and significant (\(\approx\) 16 \%
better than the MFE). The accuracy for those sequences is equivalent to ML
performances. For sequence lengths greater than 300 nucleotides, we observed the
same drastic lost of accuracy, although we took only the best prediction among
the 50 saved configurations for each sequence.

Two regions of lack of performances were observed for all methods. A group of 28
sequences of length shorter than 80 nucleotides were evaluated with free of
their known structures about 9.8 kcal/mol greater than the MFE structures. Some
of them involve large exterior loop such as displayed in \textbf{figure}. The second
region is around 200 nucleotides length. The known structure of these sequences
also displayed large unpaired regions such as the one shown in \textbf{figure}.

\begin{figure}[htbp]
\centering
\includegraphics[width=.9\linewidth]{img/fold_perf_pvv.png}
\caption{Folding comparison by taking the best energy among the 30 predicted trajectories}
\end{figure}

\begin{figure}[htbp]
\centering
\includegraphics[width=.9\linewidth]{img/comb_rna_struct.png}
\caption{Difficult structures}
\end{figure}

To investigate the region of the structure space where the thermodynamic model
tends to fail, we computed the composition content of the known structures.
\textbf{Figure} shows the prcent of base pairs or positions involved in the five loop
types: interior, exterior, hairpin, stacking, and multi-branch loops. Those
prcents were then represented in a principal component analysis. From the PCA,
we observed that the known structures are distributed in the structure space
non-uniformly. Some natural structures, as observed above, have large exterior
loops. The center of mass in the principal component space is located in between
the high density stacking and interior loops. This shows that the dataset
contains many elongated structures.

\begin{figure}[htbp]
\centering
\includegraphics[width=.9\linewidth]{img/comp_fails.png}
\caption{where does the methods failed? PCA RNAfold, Mxfold, FFT, and}
\end{figure}

The thermodynamic model tends to produce more diverse structures as shown in
\textbf{figure}. Loops content were extracted from the predicted structures of each
method and projected onto their respective two first principal components space.
Both RAFFT and MFE predictions seems to produce a diverse structure space while
the ML method does allow for long unpaired regions in long hairpins.

\begin{figure}[htbp]
\centering
\includegraphics[width=.9\linewidth]{img/content_predicted_data.png}
\caption{What kind of structure these methods naturally produced}
\end{figure}

\section{Methods}
\label{sec:org483890c}
We formed two sub-datasets based on the ArchiveII (\textbf{ref}) dataset. First, we
removed from all the structure containing pseudoknot since all tool considered
here don't handle pseudoknots. Next, we removed all the structures which were
evaluated with a positive energy or null energy with the Turner 2004 energy
parameters. Since positive energies means that the completely unfolded structure
is more stable than the native one, we assume that those structures are not well
modeled by the energy function used here. This dataset is composed of 2698
structures. 240 sequences were found multiple times (from 2 to 8 times). 19 of
them were found with different structures. We discarded all duplication and
picked the structure with the lowest energy for each. We obtained a dataset of
2296 sequences.

To compute the MFE structure, we used RNAfold (version) with the default
parameters and the Turner 2004 set of energy parameters. For the machine
learning tool, we computed the prediction using Mxfold2 with the default
parameters. The structures for both were used for the statistics.

For kinfold, we performed for each sequence, 40 simulations of 10\textsuperscript{4} (unit?).
Then, we counted the occurrences of each structures and selected the 50 most
populated structures. The best structure in terms of PPV was displayed and used
for the statistics.

For the FFT-based algorithm, we used two sets of parameters. First, we used
search for consecutive base pairs in the 50 best modes and stored 50
conformations for which we displayed the best energy found. The correlation were
computed using the weights w\textsubscript{GC}=3, w\textsubscript{AU}=2, and w\textsubscript{GU}=1.

To measure the predictions accuracy, we used two metrics from epimiology. The
positive predictive value (PPV) which is the fraction of correct base pairs
predictions in the predicted structure. The sensitivity is the fraction of
correctly predicted base pairs in the true structure. Both metrics are defined
as follow:
\begin{equation}
PPV = \frac{TP}{TP + FN} \;\;\; \text{Sensitivity} = \frac{TP}{TP+FP}
\end{equation}
where TP, FN, and FP stand respectively for the number of correctly predicted
base pairs (true positives), the number of base pairs not detected (false
negatives), and the number of wrongly predicted base pairs (false positives). To
maintain consistency with previous and future studies, we computed these metrics
using the implementation in the \texttt{scorer} tool provided in \textbf{ref Mathews}, which
provide also a more flexible estimate where shift are allowed.

The loop composition were extracted in terms of proportion to have an overall
measure of the structure distribution. We first convert all natural structures
into Shapiro notation using Vienna Package utilies. From the notation, we
extracted the proportion of base pairs involved into the interior, exterior,
bulge, stacking, and multibranch loops. For each true structure, we obtained a
prcent of type of loops from which we extracted the principal components. Next,
the structure compositions where projected on the first two principal components
for visual conveniences. The composition arrows represents the eigen vectors
obtained from the diagonalization of the covariance matrix.

\clearpage
\section{Concluding discussion}
\label{sec:org837e68f}
We have proposed a simple heuristic of the RNA folding dynamic called RAFFT.
This heuristic uses a greedy rule to fold RNAs. Groups of consecutive base pairs
found to improve the energy are formed along the procedure in such a way a
smooth and coarse grained fashion. To search for consecutive base pairs, we
implemented a FFT-based technique which takes advantage of the mirror encoding.
Once a group of base pairs are formed, the sequence is split into two un-related
segments on which one can recursively search for new group of consecutive base
pairs. For one sequence, the algorithm can follow \(k\) folding paths. Finally,
the path which leads to the structure with the lowest energy is chosen.


To assess the relevance of the folding trajectories produced, we compared the
algorithm performance for the folding task. We considered three methods to
compare with: the MFE structure computed using RNAfold, the ML estimate using
MxFold tool and the kinetic approach using kinefold. Other thermodynamic-based
and ML-based tools where investigated but not shown here. We chose the MFE since
it provide a intuitive interpretation in the structure landscape, and the MEA
prediction was not found to be significantly more accurate (\textbf{ref how bench}).

From our experiments, RAFFT had an overall performance below the MFE predictions
by \(\approx\) 10\% of PVV and <SENS> of sensitivity. The ML-based approach dominated
the predictions (70.4\% of PPV and 77.7\% of sensitivity). We observed some
drastic lost of accuracies when the known structures contained large of unpaired
regions. These regions are unlikely to be stable and assumed to be very flexible
regions which could explain their presence. However, the effect of unpaired
regions seemed less dramatic for the ML method.

The principal component analysis performed on the known structure compositions
revealed a structure spaces prone to elongated structures were large unpaired
hairpins loops and exterior loops can be observed. The PCA analysis performed on
the structures predicted by the thermodynamic-based methods (RAFFT and MFE)
shown similar structure space, where flexible loops such as long hairpins or
exterior loops are of limited number. On the other hand, the ML method seemed to
be closer to the natural structure space. According to the thermodynamic model,
those unpaired regions have a local stability equal to zero. Hence, we suppose
that those regions are actually not stable in the sens that they don't have a
unique stable structure. However, the ML-method was able to identified such
structure more consistently than thermodynamic methods. This may suggest some
overfitting effects. We argue that not being able to recover such structures
would be a proof of robustness.

Although the overall performance of RAFFT was weak compared to the state of the
art in the folding task, we observed that among the \(k=50\) predicted
trajectories, one was found to have a better accuracy than the low energy
trajectory. In fact, the gain of performance is substantial for the sequences of
length lesser than 200 nucleotides with about 16\% better in PPV than the MFE
predictions. The performance is significantly similar to the ML-base method for
that length range. Sequences of length < 200 nucleotides represent 86.4\% of the
total dataset. However, for the 140 sequences of length greater than 300
nucleotides, all \(k\) predictions per sequences were similar and performed worst
than the other methods.

Given the experiment results, we believe that RAFFT is a robust heuristic for
the folding dynamic since it can produce predictions of high accuracy for 86.4\%
of this dataset. The folding paths as calculated by RAFFT are smooth and coarse
grained since many base pairs, if it improves the energy, can be formed at once
and can lead to near-native structures. This near native coarse grained folding
path is an intuitive idea which get along with the funnel protein folding
landscape. We expect this heuristic to give valuable and complementary
information to the MFE-like predictions. However, some additional work are
necessary to determine whether the folding paths followed were experimentally
observed.

On the technical points, the mirror encoding as describe here is a versatile
tool for RNA analysis. Since it contains the relative positions of base pairs in
the whole sequence, we expect it to be extendable to other use cases such as
sequence clustering, or to the speed up of nussinov-like algorithms. On the
other hand, we are aware of the limits of chosing the maximal number of base
pairs each at each step. The greedyness of the algorithm as shown in \textbf{figure},
however, it had a limited impact on the results. We are not planning to provide
yet another folding tool, in this already crowded area of excellent softwares,
but one could combined this tool with a ML-base scoring for such a purpose.
\end{document}
